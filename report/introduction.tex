La reconnaissance faciale est une méthode de reconnaissance biométrique
qui consiste en l'identification d'un individu à partir d'une image de son
visage. Étant donné qu'un visage est unique à une personne (sauf dans des cas très
rares de jumeaux identiques où même un être humain ne peut pas faire 
la différence), la reconnaissance faciale est très adaptée pour l'identification
d'un individu.

La reconnaissance faciale a fait l'objet de plusieurs travaux de recherche ces dernières décennies,
on la retrouve aujourd'hui dans plusieurs secteurs de la vie quotidienne comme les systèmes de
contrôle d'accès, les systèmes de registre de présence, mais également dans les réseaux sociaux.

Dans ce projet, nous avons travaillé sur l'article \cite{} qui propose une méthode
basée sur les vecteurs propres des images de visages (appelés ``eigenfaces'', ce qui pourrait être 
traduit en ``visages propres'' en français. Par la suite nous allons utiliser le terme ``eigenfaces'').
C'est une méthode peu coûteuse en puissance de calcul, ce qui contraste avec les méthodes standards
actuelles qui utilisent des modèles obtenus par apprentissage profond (Deep Learning) qui nécessitent un
temps et une puissance de calcul considérables pour être appris, mais dont le taux de réussite aujourd'hui
atteint les 99\%.

Dans ce rapport, nous allons dans un premier temps expliquer la méthode proposée par \cite{} et
montrer les résultats obtenus. Ensuite, nous allons présenter notre propre implémentation de la méthode
ainsi que les résultats que nous avons obtenu. Enfin, nous allons étudier la robustesse de la méthode
au nombre de personnes, aux bruits, à l'inversion de contraste, à la rotation, et à la translation.